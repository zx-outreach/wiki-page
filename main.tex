\documentclass[a4paper, 12pt]{article}

\usepackage{tikzit}
\input{zx.tikzdefs}
\input{zx.tikzstyles}

\usepackage{hyperref}
\usepackage[utf8]{inputenc}
\usepackage{scrextend}
\usepackage[english]{babel}
\usepackage{stmaryrd}
\usepackage{graphicx}
%\usepackage{mathtools}
\usepackage{keycommand}
\usepackage{amsthm}
\usepackage{graphicx}
%For code listings
%\usepackage{minted}

\theoremstyle{definition}
\newtheorem{theorem}{Theorem}[section]
\newtheorem{corollary}[theorem]{Corollary}
\newtheorem{lemma}[theorem]{Lemma}
\newtheorem{proposition}[theorem]{Proposition}
\newtheorem{conjecture}[theorem]{Conjecture}
\newtheorem{definition}[theorem]{Definition}
\newtheorem{fact}[theorem]{Fact}
\newtheorem{example}[theorem]{Example}
\newtheorem{examples}[theorem]{Examples}
\newtheorem{example*}[theorem]{Example*}
\newtheorem{examples*}[theorem]{Examples*}
\newtheorem{remark}[theorem]{Remark}
\newtheorem{remark*}[theorem]{Remark*}
\newtheorem{question}[theorem]{Question}
\newtheorem{assumption}[theorem]{Assumption}

\newtheorem*{theorem*}{Theorem}
\newtheorem*{corollary*}{Corollary}
\newtheorem*{lemma*}{Lemma}
\newtheorem*{proposition*}{Proposition}

\begin{document}
{\Huge ZX-calculus} \\

The \textbf{ZX-calculus} is a \emph{graphical language} for reasoning about linear maps between \emph{qubits}. The calculus consists of a set of generators from which diagrams can be made, and a set of rules that govern how diagrams can be transformed into one another. The ZX-calculus is \emph{universal} in the sense that any linear map between qubits can be represented as a ZX-diagram. The standard set of rewrite rules for the ZX-calculus is \emph{complete} for the \emph{stabilizer fragment}, meaning that if two ZX-diagrams represent the same stabilizer computation, then each diagram can be rewritten into the other by application of the rules. There exist extensions to this rule-set that make the ZX-calculus complete for all linear maps between qubits.

\section{ZX-diagrams}
The building blocks of ZX-diagrams are referred to as spiders. There are two types of spiders, Z-spiders (commonly depicted as white or green), and X-spiders (commonly depicted as grey or red):
\begin{equation*}
    \tikzfig{Z-spider} \qquad \qquad \tikzfig{X-spider}
\end{equation*}
A spider can have an arbitrary amount of inputs (wires entering from the left) and outputs (exiting from the right). They can furthermore be labelled by a phase, which is a real number commonly taken to be in the interval $[-2\pi, 2\pi]$. If the phase is zero it is usually not written.

The interpretation of a spider as a linear map is given by
\begin{equation*}
    \tikzfig{Z-spider}\ := \ \ketbra{0...0}{0...0} +
e^{i \alpha} \ketbra{1...1}{1...1} \hfill
\qquad
\hfill \tikzfig{X-spider} \ := \ \ketbra{+...+}{+...+} +
e^{i \alpha} \ketbra{-...-}{-...-}
\end{equation*}
where $\ket{0}$ and $\ket{1}$ represent the standard computational basis state of a qubit, written in \emph{Dirac notation}, and $\ket{\pm} = \frac{1}{\sqrt{2}} (\ket{0} \pm \ket{1})$. Using this correspondence we can write down some well-known states and maps from quantum mechanics:
\begin{equation*}
  \begin{array}{rclcrcl}
  \tikzfig{ket-+} & = & \ket{0} + \ket{1} \ \propto \ket{+} &
  \qquad &
  \tikzfig{ket-0} & = & \ket{+} + \ket{-} \ \propto \ket{0} \\
  &\quad& & & \quad \\
  \tikzfig{Z-a} & = & \ketbra{0}{0} + e^{i \alpha} \ketbra{1}{1} =
  Z_\alpha &
  & 
  \tikzfig{X-a} & = & \ketbra{+}{+} + e^{i \alpha} \ketbra{-}{-} = X_\alpha
  \end{array}
\end{equation*}
Here, $Z_\alpha$ and $X_\alpha$ correspond to rotations over the \emph{Bloch sphere}  in the $Z$ and $X$ axis. In particular $Z_\pi$ and $X_\pi$ are the Z and X \emph{Pauli matrices}.


By juxtaposing spiders we create the \emph{tensor product} of the linear maps they represent, and by connecting wires we can compose the linear maps. 

\section{The rules of the ZX-calculus}
As the single input, single output Z-spider represents a rotation over the Z-axis in the Bloch sphere, we see that we have:
\begin{equation*}
    \tikzfig{phase-gate-compose}
\end{equation*}
Where the equality sign means that the two diagrams represent the same linear map, up to a global non-zero scalar factor. This equation can be generalised to spiders with an arbitrary amount of outgoing wires:
\begin{equation*}
    \tikzfig{Z-spider-fusion} \qquad\qquad \tikzfig{X-spider-fusion}
\end{equation*}
This is known as the \emph{spider fusion rule}. The rules of the ZX-calculus allow one to rewrite diagrams while preserving their interpretations as linear maps.

By combining the two types of spiders we can make a \emph{CNOT} gate:
\begin{equation*}
    \tikzfig{cnot}
\end{equation*}
That these different diagrams are equal is due to the meta-rule that \emph{only topology matters}. This means that when two ZX-diagrams can be transformed into one another by topological deformation, then they represent the same linear maps.

The \emph{Hadamard} gate is of special importance to the ZX-calculus, and therefore it is denoted with special notation, namely a yellow box:
\begin{equation*}
    \tikzfig{had-def}
\end{equation*}
This combination of three spiders constitutes a \emph{Euler decomposition} of the Hadamard gate, that is equal to the regular definition of the Hadamard gate up to global phase. The Hadamard gate interchanges the Z- and X-axis of the Bloch sphere. This gives rise to the \emph{colour change rule} of the ZX-calculus:
\begin{equation*}
    \tikzfig{colour-change}
\end{equation*}

We can use these representations of gates and the colour change and spider fusion rule in order to reason about quantum circuits. Consider for instance the following example of a circuit that prepares the \emph{GHZ}-state $\ket{000}+\ket{111}$:
\begin{equation*}
    \tikzfig{ghz-circuit}
\end{equation*}
We used a second rule here, which is \emph{identity removal}:
\begin{equation*}
    \tikzfig{id-removal}
\end{equation*}

The Pauli X gate has a well-known commutation relation with Z-rotations of the form $Z_\alpha X = XZ_{-\alpha}$. This is captured by $\pi$-copy rule:
\begin{equation*}
        \tikzfig{pi-copy}
\end{equation*}

Using this rule we can derive the \emph{phase-gate teleportation} schema, where we prepare a phase rotated state, and then use a CNOT and classical measurement to perform it on the qubit:
\begin{equation*}
    \tikzfig{phase-gate-teleportation}
\end{equation*}


\end{document}